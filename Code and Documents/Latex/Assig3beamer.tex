%%%%%%%%%%%%%%%%%%%%%%%%%%%%%%%%%%%%%%%%%%%%%%%%%%%%%%%%%%%%%%%
%
% Welcome to Overleaf --- just edit your LaTeX on the left,
% and we'll compile it for you on the right. If you open the
% 'Share' menu, you can invite other users to edit at the same
% time. See www.overleaf.com/learn for more info. Enjoy!
%
%%%%%%%%%%%%%%%%%%%%%%%%%%%%%%%%%%%%%%%%%%%%%%%%%%%%%%%%%%%%%%%


% Inbuilt themes in beamer
\documentclass{beamer}

% Theme choice:
\usetheme{CambridgeUS}

% Title page details: 
\title{Assignment 3}
\subtitle{\Large AI1110: Probability and Random Variables \\ \large Indian Institute of Technology Hyderabad}
\author{Rudransh Mishra}
\date{\today}
\logo{\large \LaTeX{}}


\begin{document}

% Title page frame
\begin{frame}
    \titlepage 
\end{frame}

% Remove logo from the next slides
\logo{}


% Question frame
\section{Question}
\begin{frame}{Question}
    \begin{block}{Question 8}
        A die is thrown three times,\\\\
        E : 4 appears on the third toss,\\ 
        F : 6 and 5 appears respectively on first two tosses.\\\\
        Determine P(\(\frac{E}{F}\))\\
    \end{block}
\end{frame}


% Formula frame
\begin{frame}{Formula}
    \begin{block}{Relevant formula}
        \item Probability of A occurring IF B occurs, $P(A|B)$ is given as:
            \begin{align*}
                P(A \vert B) = \frac{P(A \cap B)}{P(B)}
            \end{align*}
    \end{block}
\end{frame}
% Solution frames
\begin{frame}{Solution}
    Probability of 4 appearing on the third toss, \\
    P(E): \(\frac{1}{6}\)\\
    \\
    Probability of 6 appearing on the first toss, \\
    P1: \(\frac{1}{6}\)\\
    \\
    Probability of 5 appearing on the second toss,\\
    P2: \(\frac{1}{6}\)\\
    \\
\end{frame}
\begin{frame}{Solution(contd.)}
    Probability of 5 appearing on the second toss AND 6 appearing on the first toss, P(F) is given as:\\
\begin{align}
  &P(F)=P1\times P2
\end{align}
\begin{align}
   = \frac{1}{6} \times \frac{1}{6}
   = \frac{1}{36}
\end{align}

  Probability of E AND F, $P(E \cap F)$ is given as:

\begin{align}
  &P(E \cap F)=P(E)\times P(F)
\end{align}
\begin{align}
   = \frac{1}{6} \times \frac{1}{36}
   = \frac{1}{256}
\end{align}
\end{frame}
\begin{frame}{Solution(contd.)}
We know that,

Probability of A occurring IF B occurs, $P(A|B)$ is given as:

\begin{align}
  $P(A \vert$ B) =$ \(\frac{P(A \cap B)}{P(B)}\)$
\end{align}
$\therefore$ The probabilty of the problem being solved, P(Solved) will be given by
\begin{align}
  $P(E \vert$ F) = \(\frac{P(E \cap F)}{P(F)}\)$
\end{align}
\begin{align}
  $P(E \vert$ F) = $\(\frac{(\frac{1}{256})}{(\frac{1}{36})}\)$
\end{align}
\begin{align}
  $P(E \vert$ F) = \(\frac{1}{6}\)$
\end{align}
$\therefore$ The probabilty P(\(\frac{E}{F}\)) will be 1/6.

\end{frame}


% Python frame
\section{Python Code}
\begin{frame}{Python Code}
p1=1/6  \hspace{1.4cm} \#Probability of 4 appearing on the third toss,\\
p2=1/6  \hspace{1.4cm} \#Probability of 6 appearing on the first toss,\\
p3=1/6  \hspace{1.4cm} \#Probability of 5 appearing on the second toss\\

Pe=p1  \hspace{1.55cm} \# E is the event that 4 appears on the third toss\\
Pf=p2*p3  \hspace{1cm} \# F is the event that 6 and 5 appears respectively on first two tosses\\
PEnF=Pe*Pf \\


Pelf=PEnF/Pf  \hspace{0.4cm} \# By formula\\

print(" The probabilty P(E/F) will be ",Pelf)
\end{frame} 

\end{document}