\documentclass[journal,12pt,twocolumn]{IEEEtran}
\usepackage{enumitem}
\usepackage{amsmath}
\usepackage{amssymb}
\usepackage{graphicx}


\title{Assignment 3 \\ \Large AI1110: Probability and Random Variables \\ \large Indian Institute of Technology Hyderabad}
\author{Rudransh Mishra \\ \normalsize AI21BTECH11025 \\ \vspace*{20pt} \normalsize  9 April 2022 \\ \vspace*{20pt} \Large NCERT Grade 12}


\begin{document}
% The title
\maketitle

% The question
\textbf{Question 8} \\
A die is thrown three times,\\\\
E : 4 appears on the third toss,\\ 
F : 6 and 5 appears respectively on first two tosses.\\\\
Determine P(\(\frac{E}{F}\))\\

% The solution

\textbf{Solution.}\\
\noindent Probability of 4 appearing on the third toss, \\
P(E): \(\frac{1}{6}\)\\
Probability of 6 appearing on the first toss, \\
P1: \(\frac{1}{6}\)\\
Probability of 5 appearing on the second toss,\\
P2: \(\frac{1}{6}\)\\
\\
Probability of 5 appearing on the second toss AND 6 appearing on the first toss, P(F) is given as:\\
\begin{align}
  &P(F)=P1\times P2
\end{align}
\begin{align}
  $ = \(\frac{1}{6}\) \times \(\frac{1}{6}\)$
  $ = \(\frac{1}{36}\)$
\end{align}

  Probability of E$\cup $F, $P(E \cup F)$ is given as:

\begin{align}
  &P(E \cup F)=P(E)\times P(F)
\end{align}
\begin{align}
  $ = \(\frac{1}{6}\) \times \(\frac{1}{36}\)$
  $ = \(\frac{1}{256}\)$
\end{align}

We know that,

Probability of A occurring IF B occurs, $P(A|B)$ is given as:

\begin{align}
  $P(A \vert$ B) =$ \(\frac{P(A \cup B)}{P(B)}\)$
\end{align}
$\therefore$ The probabilty of the problem being solved, P(Solved) will be given by
\begin{align}
  $P(E \vert$ F) = \(\frac{P(E \cup F)}{P(F)}\)$
\end{align}
\begin{align}
  $P(E \vert$ F) = $\(\frac{(\frac{1}{256})}{(\frac{1}{36})}\)$
\end{align}
\begin{align}
  $P(E \vert$ F) = \(\frac{1}{6}\)$
\end{align}
$\therefore$ The probabilty P(\(\frac{E}{F}\)) will be 1/6.

\end{document}